\mytopic{Moore's Law}

In 1965, the technological progress lead Gordon E. Moore to formulate
his rule of thumb that became famous under the name ``Moore's
Law''. It states that the production costs of microchips used in
electronic circuits would fall, leading to smaller and more powerful
microchips at a lower overall cost. He estimated that the number of
components on a single chip to double roughly every 12 months.

This law has been updated several times over the following
decades. First the increase in the number of transistors used in a
single chip was the driving force behind the exponential
progress. Since the mid-1990s, and until the early 2000s, the
continuation of the trend was mostly due to the downscaling of the
components themselves. Today, as the manufacturing process is hitting
physical limits, there is a development towards integrating functions,
that until now were provided by separate components, into a single
chip.

